%-----------------------------------------------------------------------------%
\chapter{\babDua}
\thispagestyle{fancy}
%-----------------------------------------------------------------------------%

\section{Layanan Perbaikan Ponsel}
Layanan perbaikan ponsel adalah layanan teknis yang berfokus pada perbaikan perangkat keras (\textit{hardware}) ponsel yang mengalami kerusakan atau malfungsi. Perbaikan ini mencakup penggantian layar yang retak, perbaikan baterai yang melemah, penggantian \textit{port} pengisian daya yang rusak, serta perbaikan atau penggantian komponen internal seperti papan sirkuit (\textit{motherboard}), sensor, dan modul kamera \parencite{Ardhianto2020}. Teknisi menggunakan peralatan khusus untuk mendiagnosis dan mengganti komponen yang rusak guna memastikan perangkat kembali berfungsi secara optimal.

\section{Framework}
\textit{Framework} adalah kerangka kerja yang menyediakan struktur dasar untuk pengembangan perangkat lunak, yang terdiri dari pustaka kode, aturan, dan alat bantu yang mempercepat serta mempermudah proses pengembangan. \textit{Framework} bertindak sebagai fondasi standar yang memungkinkan pengembang untuk membangun aplikasi dengan lebih efisien, mengurangi kode berulang, dan meningkatkan kualitas serta keamanan perangkat lunak \parencite{Natasia2024}. \textit{Framework} dapat mencakup berbagai aspek, seperti manajemen basis data, keamanan, dan pengelolaan antarmuka pengguna, tergantung pada kebutuhan pengembangan.

\section{Next.js}
Next.Js adalah \textit{framework} berbasis React.Js yang dirancang untuk membangun aplikasi web modern dengan performa tinggi, fleksibilitas, dan pengalaman pengguna yang optimal. \textit{Framework} ini mendukung \textit{server-side rendering} (SSR), \textit{static site generation} (SSG), dan \textit{incremental static regeneration} (ISR), yang memungkinkan halaman web dimuat lebih cepat dengan pengelolaan data yang lebih efisien. Dengan fitur \textit{automatic code splitting}, Next.Js memastikan bahwa hanya kode yang dibutuhkan untuk setiap halaman yang dimuat, sehingga meningkatkan kecepatan akses. Selain itu, \textit{framework} ini memiliki \textit{routing} otomatis, yang memungkinkan pengembang membuat halaman baru hanya dengan menambahkan \textit{file} dalam direktori \textit{pages}. Next.Js juga menyediakan \textit{API Routes}, memungkinkan pengembang untuk membangun \textit{backend} dalam proyek yang sama tanpa perlu layanan tambahan. Selain itu, \textit{framework} ini memiliki optimasi gambar bawaan, yang mengurangi ukuran \textit{file} tanpa mengorbankan kualitas, serta integrasi langsung dengan berbagai layanan \textit{cloud} seperti Vercel untuk proses \textit{deployment} yang lebih mudah. Dengan kombinasi fitur-fitur ini, Next.Js menjadi pilihan utama dalam pengembangan aplikasi web yang \textit{SEO-friendly}, responsif, dan efisien, terutama bagi perusahaan yang membutuhkan solusi modern untuk meningkatkan pengalaman pengguna dan kecepatan akses \textit{website} mereka \parencite{Widodo2024}.

\section{Typescript}
TypeScript adalah bahasa pemrograman yang dikembangkan oleh Microsoft sebagai perluasan dari JavaScript, yang dirancang untuk meningkatkan skalabilitas dan keandalan dalam pengembangan perangkat lunak. TypeScript menyediakan fitur \textit{statically typed}, yang memungkinkan pengembang mendefinisikan tipe data untuk variabel, fungsi, dan objek, sehingga membantu mendeteksi kesalahan lebih awal sebelum kode dijalankan \parencite{Febryanto2024}. Dengan dukungan kompatibilitas penuh dengan JavaScript, TypeScript dapat digunakan dalam proyek yang sudah ada dan memungkinkan pengembang untuk mengadopsinya secara bertahap. Selain itu, TypeScript memiliki fitur seperti \textit{interface}, \textit{enum}, \textit{generics}, dan \textit{decorators}, yang membantu meningkatkan keterbacaan serta pemeliharaan kode dalam proyek besar. TypeScript juga sering digunakan bersama dengan \textit{framework} seperti Angular, React, dan Next.Js, serta banyak digunakan dalam pengembangan aplikasi web modern yang memerlukan manajemen kode yang lebih baik dan keamanan tipe yang lebih tinggi.

\section{Back-End}
\textit{Back-end} adalah bagian dari sistem perangkat lunak yang berfungsi sebagai inti dari aplikasi, bertanggung jawab dalam pengolahan data, logika bisnis, dan komunikasi antara server serta \textit{database}. Berbeda dengan \textit{front-end} yang menampilkan antarmuka pengguna, \textit{back-end} bekerja di belakang layar untuk menangani permintaan pengguna, melakukan perhitungan, dan mengelola sumber daya sistem. Komponen utama \textit{back-end} meliputi server, \textit{database}, dan API (\textit{Application Programming Interface}), yang memungkinkan komunikasi antara sistem \parencite{Samsudin2024}.

\section{Representational State Transfer (REST)}
\textit{Representational State Transfer} (REST) merupakan arsitektur komunikasi yang dirancang untuk pengembangan API berbasis web, memungkinkan pertukaran data antara klien dan server melalui protokol HTTP \parencite{pranata2020perancangan}. Arsitektur ini bersifat \textit{stateless}, sehingga setiap permintaan dari klien harus menyertakan seluruh informasi yang diperlukan tanpa bergantung pada penyimpanan sesi di sisi server. REST memanfaatkan metode HTTP standar seperti GET, POST, PUT, dan DELETE untuk mengelola sumber daya (\textit{resources}) secara efisien. Dengan karakteristik desain yang sederhana, ringan, dan berbasis standar terbuka, REST menjadi pilihan utama dalam pengembangan aplikasi web dan \textit{mobile} yang memprioritaskan skalabilitas, efisiensi, serta integrasi antarsistem yang \textit{seamless}.

\section{Basis Data}
Basis data adalah kumpulan informasi atau data yang terorganisir secara sistematis dan disimpan secara elektronik, sehingga mudah diakses, dikelola, dan diperbarui. Basis data digunakan untuk menyimpan berbagai jenis informasi seperti data pelanggan, transaksi, produk, atau informasi lainnya yang dibutuhkan dalam suatu sistem. Untuk mengelola basis data, digunakan perangkat lunak yang disebut \textit{Database Management System} (DBMS) seperti MySQL, PostgreSQL, dan Oracle \parencite{sulistyo2020}. Dalam pengembangan sistem informasi, basis data berperan penting sebagai pusat penyimpanan dan pengelolaan data yang mendukung kecepatan, keakuratan, dan integritas informasi.

\section{Node.Js}
Node.Js adalah platform \textit{open-source} yang memungkinkan eksekusi kode JavaScript di luar peramban (\textit{browser}), khususnya di sisi server. Dengan menggunakan arsitektur \textit{event-driven} dan \textit{non-blocking I/O}, Node.Js sangat efisien untuk membangun aplikasi jaringan berskala besar seperti \textit{chat real-time}, \textit{streaming}, dan layanan berbasis API \parencite{Balino2024}. Node.Js menggunakan satu \textit{thread} untuk menangani banyak permintaan secara bersamaan, yang membuatnya ringan dan cepat dalam memproses beban kerja tinggi. Kemampuan ini menjadikan Node.Js populer di kalangan pengembang yang ingin membangun aplikasi web modern dengan performa tinggi menggunakan satu bahasa pemrograman dari sisi klien hingga server.

\section{PostgreSQL}
PostgreSQL adalah sistem manajemen basis data relasional (RDBMS) \textit{open-source} yang kuat, andal, dan kaya fitur, yang digunakan secara luas dalam pengembangan aplikasi skala kecil hingga besar \parencite{Boiko2025}. PostgreSQL mendukung SQL standar dan berbagai fitur lanjutan seperti transaksi ACID, \textit{foreign key}, \textit{indexing}, \textit{view}, serta \textit{stored procedure}. Selain itu, PostgreSQL juga mendukung data non-relasional seperti JSON dan XML, menjadikannya fleksibel untuk kebutuhan modern seperti aplikasi web dan sistem analitik. Kemampuannya untuk menangani beban kerja yang kompleks dan skalabilitas tinggi menjadikannya pilihan populer di kalangan pengembang dan perusahaan teknologi.

\section{Object Relational Mapping (ORM)}
\textit{Object Relational Mapping} (ORM) merupakan teknik pemrograman yang bertujuan menjembatani perbedaan antara objek dalam bahasa pemrograman berorientasi objek (OOP) dan struktur tabel dalam basis data relasional. Melalui ORM, pengembang dapat mengakses dan memanipulasi data basis data melalui operasi objek tanpa perlu menulis perintah SQL secara eksplisit, sehingga mempercepat pengembangan aplikasi. Konsep ini memetakan entitas seperti tabel \textit{database} ke dalam kelas (\textit{class}) dan baris data ke dalam objek (\textit{object}), sehingga struktur data dalam kode menjadi konsisten dengan skema basis data \parencite{riyanto2021implementasi}.
ORM juga mendukung fitur lanjutan seperti validasi data otomatis, manajemen relasi antar-tabel (\textit{foreign key}), dan migrasi skema basis data, yang meminimalkan kesalahan konversi format data. \textit{Framework} ORM seperti Sequelize (JavaScript), SQLAlchemy (Python), dan Hibernate (Java) menjadi pilihan populer karena kemampuannya dalam mempercepat pengembangan aplikasi berbasis \textit{database} dengan pendekatan \textit{object-oriented}.

\section{Agile}
Metode Agile merupakan pendekatan pengembangan perangkat lunak yang bersifat iteratif dan inkremental, dengan fokus pada kolaborasi tim, fleksibilitas terhadap perubahan, serta keterlibatan aktif pengguna atau pelanggan selama proses pengembangan. Proyek dibagi menjadi siklus pendek (\textit{sprint}), di mana setiap iterasi menghasilkan bagian produk yang dapat diuji dan dievaluasi langsung, memastikan produk sesuai kebutuhan \parencite{fajri2024analisis}. Agile mengedepankan prinsip transparansi, komunikasi terbuka, dan umpan balik cepat melalui evaluasi berkala (\textit{retrospective}), sehingga perubahan dan penyesuaian dapat dilakukan secara dinamis. Metode ini tidak hanya meningkatkan efisiensi dalam skala tim, tetapi juga bermanfaat bagi pengembang individu dengan mendorong adaptasi terhadap perubahan, disiplin melalui perencanaan iteratif, dan peningkatan keterampilan komunikasi serta kerja sama. Agile juga mendorong belajar berkelanjutan melalui analisis hasil \textit{sprint} sebelumnya, peningkatan proses kerja, serta peningkatan produktivitas dan kualitas pengembangan secara bertahap.

\section{\textit{Progressive Web App} (PWA)}
\textit{Progressive Web App} (PWA) merupakan jenis aplikasi web \textit{modern} yang dibangun menggunakan standar web seperti HTML, CSS, dan JavaScript, namun memberikan pengalaman pengguna yang menyerupai aplikasi \textit{native}, termasuk kemampuan berjalan secara \textit{offline} melalui \textit{service worker} dan instalasi langsung dari \textit{browser} ke \textit{home screen} perangkat. Dalam konteks akademik, PWA juga dilaporkan mampu memperkuat keterlibatan pengguna (\textit{user engagement}) melalui fitur seperti notifikasi \textit{push} dan \textit{background synchronization}, serta peningkatan performa seperti waktu muat yang cepat dan interaktivitas yang tinggi. Selain itu, penelitian implementasi PWA pada berbagai sistem informasi (seperti situs program studi, platform \textit{volunteer}, dan layanan GetHelp berbasis Next.js) menunjukkan bahwa PWA berkontribusi pada pengalaman penggunaan yang lebih responsif, stabil, dan dapat diakses lintas perangkat \cite{Apriyanti2025PWA_PRODI_TI}.

\section{Use Case Diagram}
\textit{Use Case Diagram} adalah salah satu jenis diagram dalam \textit{Unified Modeling Language} (UML) yang digunakan untuk menggambarkan interaksi antara aktor (pengguna atau sistem lain) dengan sistem yang sedang dikembangkan, berdasarkan fungsionalitas yang disediakan oleh sistem tersebut \parencite{friadi2023perancangan}. Diagram ini menampilkan hubungan antara aktor dan \textit{use case} (fitur atau layanan sistem) dalam bentuk visual, sehingga memudahkan pemahaman mengenai kebutuhan sistem dari sudut pandang pengguna. \textit{Use case diagram} membantu pengembang, analis, dan pemangku kepentingan untuk memahami ruang lingkup sistem, mengidentifikasi fitur utama, serta merancang arsitektur sistem secara efektif sejak tahap awal pengembangan. Dengan menyajikan gambaran sederhana namun menyeluruh tentang fungsi-fungsi sistem, diagram ini menjadi alat komunikasi yang kuat antara tim teknis dan non-teknis.

\section{Activity Diagram}
\textit{Activity Diagram} adalah jenis diagram dalam \textit{Unified Modeling Language} (UML) yang digunakan untuk memodelkan alur kerja atau proses bisnis dalam suatu sistem, baik secara sistematis maupun logis. Diagram ini menggambarkan urutan aktivitas, keputusan, percabangan, serta kondisi awal dan akhir dari suatu proses dengan jelas dan terstruktur. \textit{Activity diagram} sangat berguna untuk memahami bagaimana suatu proses berjalan dari satu aktivitas ke aktivitas lainnya, termasuk kondisi pengambilan keputusan dan kemungkinan paralelisme aktivitas. Dalam konteks pengembangan perangkat lunak, diagram ini membantu pengembang dan analis sistem dalam merancang logika proses, mendeteksi potensi permasalahan, dan mengomunikasikan alur proses kepada pemangku kepentingan secara visual \parencite{septiansyah2024sistem}.

\section{Black Box Testing}
\textit{Black Box Testing} merupakan metode pengujian perangkat lunak yang berfokus pada fungsi eksternal sistem tanpa melibatkan struktur internal atau kode program. Dalam pendekatan ini, penguji hanya mengetahui \textit{input} yang diberikan dan \textit{output} yang diharapkan, tanpa pengetahuan mengenai proses algoritma atau logika implementasi di dalam sistem. Tujuan utama \textit{Black Box Testing} adalah memastikan bahwa perangkat lunak beroperasi sesuai dengan spesifikasi fungsional dan kebutuhan pengguna, termasuk validasi fungsi, antarmuka, serta performa sistem. Metode ini efektif untuk mengidentifikasi kesalahan seperti ketidaksesuaian \textit{input-output}, \textit{bug} pada antarmuka pengguna, atau respons yang tidak sesuai dengan ekspektasi pengguna \parencite{fikri2024pengujian}.

\section{User Experience Questionnaire (UEQ)}
\textit{User Experience Questionnaire} (UEQ) merupakan alat evaluasi standar yang dirancang untuk mengukur kualitas pengalaman pengguna (\textit{user experience}) terhadap produk digital seperti aplikasi, sistem informasi, atau \textit{website}. UEQ mengumpulkan persepsi subjektif pengguna melalui 26 item pertanyaan yang disusun dalam bentuk pasangan kata bipolar (misalnya: rumit – mudah , membosankan – menarik) dinilai menggunakan skala Likert 7 poin \parencite{Savitri2023}. Item ini dikelompokkan dalam enam skala utama:
\begin{enumerate}
\item \textit{Attractiveness} (daya tarik secara umum),
\item \textit{Perspicuity} (kemudahan pemahaman),
\item \textit{Efficiency} (efisiensi penggunaan),
\item \textit{Dependability} (keandalan),
\item \textit{Stimulation} (tingkat keterlibatan pengguna), dan
\item \textit{Novelty} (kebaruan atau kreativitas).
\end{enumerate}
Dalam praktik, UEQ menjadi alat penting untuk mengevaluasi kualitas antarmuka (\textit{user interface}) dan interaksi sistem dari perspektif pengguna akhir. Hasil analisis UEQ memberikan wawasan untuk:
\begin{itemize} 
\item Mengidentifikasi kelebihan dan kelemahan sistem berdasarkan persepsi pengguna,
\item Melakukan perbaikan berbasis data untuk meningkatkan kualitas pengalaman pengguna,
\item Membandingkan kinerja antar produk atau versi sistem secara objektif.
\end{itemize}
UEQ juga mendukung kebutuhan internasional karena tersedia dalam lebih dari 30 bahasa, termasuk bahasa Indonesia, sehingga memudahkan implementasi di berbagai konteks. Daftar pertanyaan dapat dilihat pada Gambar \ref{fig:ueq}.
\begin{figure}[H]
    \centering
    \includegraphics[width=0.9\linewidth]{assets/images/bab2/ueq.jpg}
    \caption{Daftar Pertanyaan Metode UEQ}
    \label{fig:ueq}
\end{figure}

\section{Penelitian Terkait}
Penelitian oleh Dihki Ardhianto pada tahun 2020 dengan judul "Aplikasi Jasa Servis Handphone Berbasis Web (Studi Kasus: Toko Teknisi Tamvan Gunungkidul)" bertujuan untuk membangun aplikasi jasa servis \textit{handphone} berbasis web pada Toko Teknisi Tamvan di Gunungkidul, yang sebelumnya masih menggunakan sistem manual dalam pelayanan dan pencatatan servis serta penjualan \textit{sparepart}. Sistem manual ini menyebabkan keterbatasan informasi bagi pelanggan mengenai status perbaikan \textit{handphone}, serta menyulitkan teknisi dalam mengelola data servis dan persediaan barang. Aplikasi yang dikembangkan menyediakan fitur pemesanan jasa servis, pembelian \textit{sparepart}, pengecekan status perbaikan, dan pengelolaan laporan, sehingga memudahkan proses administrasi serta meningkatkan efisiensi operasional toko. Metode yang digunakan dalam penelitian ini meliputi tahapan analisis kebutuhan sistem, perancangan sistem dengan pendekatan \textit{Data Flow Diagram} (DFD) dan \textit{Entity Relationship Diagram} (ERD), implementasi menggunakan teknologi berbasis web (HTML, PHP, JavaScript, MySQL), serta pengujian fungsional terhadap aplikasi. Hasil dari implementasi menunjukkan bahwa aplikasi yang dibangun dapat mempercepat penyimpanan dan pengelolaan data servis maupun penjualan \textit{sparepart}, memberikan notifikasi kepada pelanggan, menyederhanakan pembuatan laporan, dan meningkatkan kontrol terhadap persediaan barang. Penelitian ini menyimpulkan bahwa sistem informasi berbasis web dapat menjadi solusi efektif dalam meningkatkan pelayanan dan kinerja operasional usaha jasa servis \textit{handphone} \parencite{Ardhianto2020}.
Penelitian oleh Ahmad Rizal, M. Dedy Rosyadi, dan Muhammad Amin pada tahun 2022 dengan judul "Rancang Bangun Aplikasi Service Smartphone Berbasis Android" bertujuan mengembangkan aplikasi \textit{mobile} berbasis Android untuk memfasilitasi pemesanan layanan servis \textit{smartphone} dan penjualan \textit{sparepart} secara \textit{online} pada toko Phone Comp Service. Aplikasi ini dirancang untuk menggantikan sistem manual dengan memungkinkan konsumen memantau status perbaikan, melakukan pemesanan, dan mengakses informasi layanan tanpa harus datang ke toko. Metode penelitian mencakup observasi, wawancara, studi pustaka, serta pengembangan sistem menggunakan model Waterfall dengan teknologi Android Studio, Framework7, Firebase sebagai \textit{database}, dan PHP sebagai web server. Hasil penelitian menunjukkan aplikasi berhasil meningkatkan efisiensi waktu konsumen melalui fitur pemesanan \textit{online}, pelacakan status servis, serta laporan manajemen \textit{sparepart}. Selain itu, aplikasi juga berkontribusi pada peningkatan omset toko dengan memperluas jangkauan pemasaran secara digital \parencite{rizal2022rancang}. Perbedaan penelitian dapat dilihat pada Tabel \ref{tab:perbedaan-penelitian}.

\begin{table}[H]
\centering
\caption{Perbedaan Penelitian}
\label{tab:perbedaan-penelitian}
\resizebox{\textwidth}{!}{%
\begin{tabular}{|
>{\columncolor[HTML]{EFEFEF}}c |c|c|c|}
\hline
\textbf{Aspek} & \cellcolor[HTML]{EFEFEF}\textbf{\begin{tabular}[c]{@{}c@{}}Aplikasi Jasa Servis \\ Handphone Berbasis Web\\ (Toko Teknisi Tamvan)\end{tabular}} & \cellcolor[HTML]{EFEFEF}\textbf{\begin{tabular}[c]{@{}c@{}}Rancang Bangun Aplikasi \\ Service Smartphone \\ Berbasis Android\end{tabular}} & \cellcolor[HTML]{EFEFEF}\textbf{\begin{tabular}[c]{@{}c@{}}Rancang-Bangun Sistem \\  Informasi Pemesanan Pada Bukhari \\ Service Center Berbasis Web\end{tabular}}\\ 

\hline
\textbf{Teknologi} & \begin{tabular}[c]{@{}c@{}} HTML, PHP, \\ JavaScript, MySQL \end{tabular} & \begin{tabular}[c]{@{}c@{}} Android Studio (Framework7), \\ Firebase, PHP \end{tabular} & \begin{tabular}[c]{@{}c@{}} Next.js (React) + PostgreSQL + \\ Progessive Web App (PWA) \end{tabular} \\ 

\hline
\textbf{Fitur Utama} & \begin{tabular}[c]{@{}c@{}}Penyimpanan data servis \\ dan sparepart, Notifikasi status \\ servis, dan Laporan penjualan \end{tabular} & \begin{tabular}[c]{@{}c@{}} Pemesanan online via aplikasi \\ Android, Tracking status perbaikan, \\ dan Integrasi pembayaran \end{tabular} & \begin{tabular}[c]{@{}c@{}} Booking service dengan input \\ kerusakan, Notifikasi via email, \\ dan Biaya transparan. \end{tabular} \\ 

\hline
\textbf{Keunggulan} & \begin{tabular}[c]{@{}c@{}}Sistem terstruktur untuk \\ manajemen data dan \\ Tersedia laporan otomatis\end{tabular} & \begin{tabular}[c]{@{}c@{}}Aplikasi mobile untuk efisiensi \\ waktu dan Integrasi Firebase \\ untuk real-time data \end{tabular} & \begin{tabular}[c]{@{}c@{}}Transparansi biaya dan \\ Fleksibilitas lokasi (mekanik \\ datang ke pelanggan)\end{tabular} \\ 

\hline
\textbf{Metode Pengembangan} & \begin{tabular}[c]{@{}c@{}}Analisis kebutuhan, DFD, \\ ERD\end{tabular} & \begin{tabular}[c]{@{}c@{}}Waterfall (Observasi, Wawancara, \\ Studi Pustaka)\end{tabular} & \begin{tabular}[c]{@{}c@{}}Agile (Analisis kebutuhan, \\ ERD, UML) + Integrasi \\ API untuk notifikasi\end{tabular} \\ \hline
\textbf{Basis Data} & MySQL & Firebase & PostgreSQ \\ 

\hline
\end{tabular}
}
\end{table}