%-----------------------------------------------------------------------------%
\chapter{\babTiga}
\thispagestyle{fancy}
%-----------------------------------------------------------------------------%


%-----------------------------------------------------------------------------%
\section{Waktu dan Lokasi Penelitian}
Penelitian ini akan bertempat di Ruang Lab Sistem Informasi dan \textit{Database}. Waktu yang dibutuhkan agar penelitian ini dapat diimplementasikan adalah 6 bulan terhitung dari bulan Juni 2025 hingga November 2025. Adapun rincian jadwal pelaksanaan penelitian dapat dilihat pada Tabel \ref{tab:jadwal_pelaksanaan}.


\begin{table}[H]
\centering
\caption{Jadwal Pelaksanaan Penelitian}
\label{tab:jadwal_pelaksanaan}
\resizebox{\textwidth}{!}{%
\begin{tabular}{|l|l|llllllllllllllllllllllll|}
\hline
&
&
\multicolumn{24}{c|}{\textbf{Bulan Ke -}} \\ \cline{3-26}
&
&
\multicolumn{4}{c|}{\textbf{Juni}} &
\multicolumn{4}{c|}{\textbf{Juli}} &
\multicolumn{4}{c|}{\textbf{Agustus}} &
\multicolumn{4}{c|}{\textbf{September}} &
\multicolumn{4}{c|}{\textbf{Oktober}} &
\multicolumn{4}{c|}{\textbf{November}} \\ \cline{3-26}
\multirow{3}{*}{\textbf{No}} &
\multirow{3}{*}{\textbf{Kegiatan}} &
\multicolumn{1}{c|}{\textbf{1}} &
\multicolumn{1}{c|}{\textbf{2}} &
\multicolumn{1}{c|}{\textbf{3}} &
\multicolumn{1}{c|}{\textbf{4}} &
\multicolumn{1}{c|}{\textbf{1}} &
\multicolumn{1}{c|}{\textbf{2}} &
\multicolumn{1}{c|}{\textbf{3}} &
\multicolumn{1}{c|}{\textbf{4}} &
\multicolumn{1}{c|}{\textbf{1}} &
\multicolumn{1}{c|}{\textbf{2}} &
\multicolumn{1}{c|}{\textbf{3}} &
\multicolumn{1}{c|}{\textbf{4}} &
\multicolumn{1}{c|}{\textbf{1}} &
\multicolumn{1}{c|}{\textbf{2}} &
\multicolumn{1}{c|}{\textbf{3}} &
\multicolumn{1}{c|}{\textbf{4}} &
\multicolumn{1}{c|}{\textbf{1}} &
\multicolumn{1}{c|}{\textbf{2}} &
\multicolumn{1}{c|}{\textbf{3}} &
\multicolumn{1}{c|}{\textbf{4}} &
\multicolumn{1}{c|}{\textbf{1}} &
\multicolumn{1}{c|}{\textbf{2}} &
\multicolumn{1}{c|}{\textbf{3}} &
\multicolumn{1}{c|}{\textbf{4}} \\ \hline

1&
Penyusunan Proposal &
\multicolumn{1}{l|}{} &
\multicolumn{1}{l|}{} &
\multicolumn{1}{l|}{\cellcolor[HTML]{333333}} &
\multicolumn{1}{l|}{\cellcolor[HTML]{333333}} &
\multicolumn{1}{l|}{} &
\multicolumn{1}{l|}{} &
\multicolumn{1}{l|}{} &
\multicolumn{1}{l|}{} &
\multicolumn{1}{l|}{} &
\multicolumn{1}{l|}{} &
\multicolumn{1}{l|}{} &
\multicolumn{1}{l|}{} &
\multicolumn{1}{l|}{} &
\multicolumn{1}{l|}{} &
\multicolumn{1}{l|}{} &
\multicolumn{1}{l|}{} &
\multicolumn{1}{l|}{} &
\multicolumn{1}{l|}{} &
\multicolumn{1}{l|}{} &
\multicolumn{1}{l|}{} &
\multicolumn{1}{l|}{} &
\multicolumn{1}{l|}{} &
\multicolumn{1}{l|}{} &
\multicolumn{1}{l|}{} \\ \hline

2&
Identifikasi Masalah &
\multicolumn{1}{l|}{} &
\multicolumn{1}{l|}{} &
\multicolumn{1}{l|}{} &
\multicolumn{1}{l|}{} &
\multicolumn{1}{l|}{\cellcolor[HTML]{333333}} &
\multicolumn{1}{l|}{\cellcolor[HTML]{333333}} &
\multicolumn{1}{l|}{} &
\multicolumn{1}{l|}{} &
\multicolumn{1}{l|}{} &
\multicolumn{1}{l|}{} &
\multicolumn{1}{l|}{} &
\multicolumn{1}{l|}{} &
\multicolumn{1}{l|}{} &
\multicolumn{1}{l|}{} &
\multicolumn{1}{l|}{} &
\multicolumn{1}{l|}{} &
\multicolumn{1}{l|}{} &
\multicolumn{1}{l|}{} &
\multicolumn{1}{l|}{} &
\multicolumn{1}{l|}{} &
\multicolumn{1}{l|}{} &
\multicolumn{1}{l|}{} &
\multicolumn{1}{l|}{} &
\multicolumn{1}{l|}{} \\ \hline

3&
Studi Literatur &
\multicolumn{1}{l|}{} &
\multicolumn{1}{l|}{} &
\multicolumn{1}{l|}{} &
\multicolumn{1}{l|}{} &
\multicolumn{1}{l|}{\cellcolor[HTML]{333333}} &
\multicolumn{1}{l|}{\cellcolor[HTML]{333333}} &
\multicolumn{1}{l|}{} &
\multicolumn{1}{l|}{} &
\multicolumn{1}{l|}{} &
\multicolumn{1}{l|}{} &
\multicolumn{1}{l|}{} &
\multicolumn{1}{l|}{} &
\multicolumn{1}{l|}{} &
\multicolumn{1}{l|}{} &
\multicolumn{1}{l|}{} &
\multicolumn{1}{l|}{} &
\multicolumn{1}{l|}{} &
\multicolumn{1}{l|}{} &
\multicolumn{1}{l|}{} &
\multicolumn{1}{l|}{} &
\multicolumn{1}{l|}{} &
\multicolumn{1}{l|}{} &
\multicolumn{1}{l|}{} &
\multicolumn{1}{l|}{} \\ \hline

4&
Analisis Kebutuhan &
\multicolumn{1}{l|}{} &
\multicolumn{1}{l|}{} &
\multicolumn{1}{l|}{} &
\multicolumn{1}{l|}{} &
\multicolumn{1}{l|}{} &
\multicolumn{1}{l|}{} &
\multicolumn{1}{l|}{\cellcolor[HTML]{333333}} &
\multicolumn{1}{l|}{\cellcolor[HTML]{333333}} &
\multicolumn{1}{l|}{\cellcolor[HTML]{333333}} &
\multicolumn{1}{l|}{} &
\multicolumn{1}{l|}{} &
\multicolumn{1}{l|}{} &
\multicolumn{1}{l|}{} &
\multicolumn{1}{l|}{} &
\multicolumn{1}{l|}{} &
\multicolumn{1}{l|}{} &
\multicolumn{1}{l|}{} &
\multicolumn{1}{l|}{} &
\multicolumn{1}{l|}{} &
\multicolumn{1}{l|}{} &
\multicolumn{1}{l|}{} &
\multicolumn{1}{l|}{} &
\multicolumn{1}{l|}{} &
\multicolumn{1}{l|}{} \\ \hline

5&
Perancangan Sistem &
\multicolumn{1}{l|}{} &
\multicolumn{1}{l|}{} &
\multicolumn{1}{l|}{} &
\multicolumn{1}{l|}{} &
\multicolumn{1}{l|}{} &
\multicolumn{1}{l|}{} &
\multicolumn{1}{l|}{} &
\multicolumn{1}{l|}{} &
\multicolumn{1}{l|}{\cellcolor[HTML]{333333}} &
\multicolumn{1}{l|}{\cellcolor[HTML]{333333}} &
\multicolumn{1}{l|}{\cellcolor[HTML]{333333}} &
\multicolumn{1}{l|}{\cellcolor[HTML]{333333}} &
\multicolumn{1}{l|}{} &
\multicolumn{1}{l|}{} &
\multicolumn{1}{l|}{} &
\multicolumn{1}{l|}{} &
\multicolumn{1}{l|}{} &
\multicolumn{1}{l|}{} &
\multicolumn{1}{l|}{} &
\multicolumn{1}{l|}{} &
\multicolumn{1}{l|}{} &
\multicolumn{1}{l|}{} &
\multicolumn{1}{l|}{} &
\multicolumn{1}{l|}{} \\ \hline

6&
Pengembangan Sistem &
\multicolumn{1}{l|}{} &
\multicolumn{1}{l|}{} &
\multicolumn{1}{l|}{} &
\multicolumn{1}{l|}{} &
\multicolumn{1}{l|}{} &
\multicolumn{1}{l|}{} &
\multicolumn{1}{l|}{} &
\multicolumn{1}{l|}{} &
\multicolumn{1}{l|}{} &
\multicolumn{1}{l|}{} &
\multicolumn{1}{l|}{} &
\multicolumn{1}{l|}{\cellcolor[HTML]{333333}} &
\multicolumn{1}{l|}{\cellcolor[HTML]{333333}} &
\multicolumn{1}{l|}{\cellcolor[HTML]{333333}} &
\multicolumn{1}{l|}{\cellcolor[HTML]{333333}} &
\multicolumn{1}{l|}{\cellcolor[HTML]{333333}} &
\multicolumn{1}{l|}{\cellcolor[HTML]{333333}} &
\multicolumn{1}{l|}{\cellcolor[HTML]{333333}} &
\multicolumn{1}{l|}{\cellcolor[HTML]{333333}} &
\multicolumn{1}{l|}{} &
\multicolumn{1}{l|}{} &
\multicolumn{1}{l|}{} &
\multicolumn{1}{l|}{} &
\multicolumn{1}{l|}{} \\ \hline

7&
Pengujian dan Evaluasi &
\multicolumn{1}{l|}{} &
\multicolumn{1}{l|}{} &
\multicolumn{1}{l|}{} &
\multicolumn{1}{l|}{} &
\multicolumn{1}{l|}{} &
\multicolumn{1}{l|}{} &
\multicolumn{1}{l|}{} &
\multicolumn{1}{l|}{} &
\multicolumn{1}{l|}{} &
\multicolumn{1}{l|}{} &
\multicolumn{1}{l|}{} &
\multicolumn{1}{l|}{} &
\multicolumn{1}{l|}{} &
\multicolumn{1}{l|}{} &
\multicolumn{1}{l|}{} &
\multicolumn{1}{l|}{} &
\multicolumn{1}{l|}{} &
\multicolumn{1}{l|}{} &
\multicolumn{1}{l|}{} &
\multicolumn{1}{l|}{\cellcolor[HTML]{333333}} &
\multicolumn{1}{l|}{\cellcolor[HTML]{333333}} &
\multicolumn{1}{l|}{} &
\multicolumn{1}{l|}{} &
\multicolumn{1}{l|}{} \\ \hline
\end{tabular}%
}
\end{table}

\section{Alat dan Bahan} 
\par Alat dan Bahan yang akan digunakan pada penelitian ini terdiri dari beberapa perangkat keras (\textit{hardware}) dan perangkat lunak (\textit{software}) yang dijabarkan sebagai berikut:
\begin{enumerate}
    \item Perangkat Keras
        \begin{itemize}
            \item Perangkat keras yang digunakan dalam penelitian ini adalah satu unit Lenovo Legion 5 Pro    dengan prosesor AMD Ryzen 7 5800H, memori 16 GBRAM,danpenyimpanan 1 TB SSD.
        \end{itemize}
    \item Perangkat Lunak
        \begin{itemize}
            \item Google Chrome
            \item Figma
            \item Cursor IDE
            \item Next.Js v13
            \item Material UI v5.16.7
            \item Typescript v5.6
            \item Prisma v5.1.1
            \item Postman v11.20.0
            \item Node.Js v20.10.0
            \item DbDiagram.io
            \item PostgreSQL
        \end{itemize}
\end{enumerate}

\section{Alur Penelitian}
\par Metode penelitian yang dilakukan pada penelitian terdiri dari pada beberapa
tahapan sebagaimana ditunjukkan pada Gambar \ref{alur_penelitian}.

\begin{figure}[H]
\centering
{\includegraphics [width = 1\textwidth]{assets/images/bab3/diagramalurpenelitian.png}}
\caption{Diagram Alur Penelitian}
\label{alur_penelitian}
\end{figure}

\subsection{Identifikasi Masalah}
\par Berdasarkan 
\par Penelitian ini dimulai dengan mengidentifikasi masalah sebagai langkah awal untuk memahami isu yang dihadapi, yang menjadi dasar pelaksanaan penelitian. Identifikasi dilakukan melalui wawancara langsung terhadap pihak terkait yaitu di Bukhari Service Center dan membandingkannya dengan praktik umum di bengkel servis \textit{handphone} lainnya di Banda Aceh. Hasil wawancara menunjukkan bahwa Bukhari Service Center memiliki keunggulan dalam hal layanan pemanggilan teknisi ke lokasi pelanggan serta transparansi dalam biaya servis dan \textit{sparepart}. Dari hasil identifikasi berdasarkan wawancara tersebut, ditemukan beberapa permasalahan utama sebagai berikut:
\begin{enumerate}
    \item Pelanggan harus datang langsung ke bengkel untuk mengetahui status perbaikan, yang menyebabkan pemborosan waktu dan biaya.
    \item Belum ada transparansi terkait rincian harga sparepart dan biaya servis, sehingga pelanggan sulit mengetahui estimasi biaya sebelum perbaikan, yang menimbulkan ketidakpastian dan berpotensi mengurangi kepercayaan terhadap layanan.
    \item Kesulitan memperoleh pembaruan terbaru terkait status perbaikan karena komunikasi masih dilakukan secara manual ke tempat servis.
\end{enumerate}
\par Berdasarkan identifikasi masalah yang telah dijabarkan, dibutuhkan sebuah sistem informasi pemesanan berbasis web yang mampu memberikan kemudahan bagi pelanggan dalam melakukan pemesanan layanan, memantau status servis, menerima notifikasi secara otomatis, dan memperoleh transparansi informasi biaya, sehingga dapat meningkatkan efisiensi pelayanan dan mendukung pengelolaan operasional Bukhari Service Center secara lebih modern dan profesional.

\subsection{Studi Literatur}
\par Pada tahap studi literatur, dilakukan proses pencarian, pengumpulan, dan analisis berbagai sumber referensi yang relevan guna memperkuat landasan teori serta memahami perkembangan teknologi dan pendekatan yang sesuai dengan topik penelitian ini. Studi literatur bertujuan untuk memperoleh gambaran yang jelas mengenai permasalahan yang diteliti, metode yang digunakan oleh peneliti sebelumnya, serta solusi yang telah diterapkan dalam kasus serupa. Dalam proses ini, peneliti meninjau baik sumber referensi berupa artikel ilmiah maupun dokumentasi teknis dari situs resmi yang mendukung pengembangan sistem informasi pemesanan berbasis web untuk layanan servis \textit{handphone}. Berikut adalah beberapa referensi yang telah ditelaah dalam penelitian ini:
\begin{enumerate}
    \item Website:
    \begin{itemize}
        \item Apple Support
    \end{itemize}
    \item Jurnal Ilmiah
    \begin{itemize}
        \item Aplikasi Jasa Servis Handphone Berbasis Web
        \item Rancang Bangun Aplikasi Service Smartphone Berbasis Android 
    \end{itemize}
\end{enumerate}

\subsection{Analisis Kebutuhan}
\par Tahap analisis kebutuhan dilakukan untuk merumuskan dan memahami secara rinci apa saja yang diperlukan oleh sistem agar dapat berjalan sesuai dengan tujuan yang diharapkan. Proses ini bertujuan untuk mengidentifikasi fitur-fitur penting, informasi yang harus dikelola, serta interaksi antara pengguna dan sistem. Informasi kebutuhan diperoleh melalui proses observasi terhadap operasional di Bukhari Service Center dan komunikasi dengan pihak terkait, sehingga sistem yang dirancang nantinya dapat menyelesaikan permasalahan yang ada. Berdasarkan hasil analisis tersebut, berikut hasil yang diperoleh:

\begin{enumerate}
    \item \textbf{Identifikasi Pengguna}
    \begin{enumerate}
        \item Pelanggan
        \begin{itemize}
            \item Melakukan pendaftaran akun.
            \item Melakukan autentikasi sebagai pelanggan.
            \item Memilih layanan servis yang tersedia.
            \item Melihat estimasi biaya servis dan harga \textit{sparepart} secara transparan.
            \item Memesan layanan servis, baik dengan mengunjungi toko atau meminta teknisi datang ke lokasi.
            \item Memilih jadwal kedatangan teknisi sesuai waktu yang tersedia.
            \item Melihat status perbaikan handphone secara \textit{real-time}.
            \item Menerima notifikasi melalui email yang berisi rincian biaya dan perkembangan status servis.
            \item Melihat riwayat layanan servis yang pernah dilakukan.
            \item Membatalkan pesanan layanan apabila diperlukan.
        \end{itemize}
        \item Admin
        \begin{itemize}
            \item Melakukan autentikasi sebagai admin.
            \item Mengelola data pelanggan (menambah, mengedit, dan menghapus).
            \item Melihat dan mengelola daftar pemesanan layanan dari pelanggan.
            \item Memverifikasi dan memperbarui status servis secara berkala.
            \item Mengatur jadwal kedatangan teknisi ke lokasi pelanggan.
            \item Mengelola laporan servis dan transaksi.
            \item Mengirimkan notifikasi email kepada pelanggan terkait \textit{update} layanan.
        \end{itemize}
    \end{enumerate}
    \item \textbf{Identifikasi Kebutuhan Fungsional}
    \par Berdasarkan analisis kebutuhan pengguna, sistem informasi pemesanan pada Bukhari Service Center harus memenuhi kebutuhan fungsional yang dikelompokkan sebagai berikut:
    \begin{itemize}
        \item \textbf{Manajemen Pengguna}: Sistem menyediakan fitur registrasi dan autentikasi untuk pelanggan dan admin dengan menggunakan email dan \textit{password}.
        \item \textbf{Manajemen Pemesanan}: Sistem memfasilitasi proses pemesanan layanan servis secara \textit{online} dengan opsi pemilihan jenis layanan (datang ke toko atau teknisi ke lokasi), pemilihan jadwal, input detail kerusakan perangkat, serta fitur pembatalan pemesanan.
        \item \textbf{Monitoring dan Notifikasi}: Sistem dapat menampilkan status perbaikan secara \textit{real-time}, mengirimkan notifikasi email otomatis untuk setiap perubahan status, serta menyediakan transparansi informasi biaya \textit{sparepart} dan jasa perbaikan.
        \item \textbf{Manajemen Data Admin}: Sistem memungkinkan admin untuk mengelola data pelanggan, memverifikasi dan memperbarui status servis, mengatur jadwal teknisi, serta menghasilkan laporan servis dan transaksi.
        \item \textbf{Riwayat Layanan}: Sistem dapat mencatat dan menampilkan riwayat layanan servis pelanggan yang dapat diakses kapan saja untuk referensi layanan berikutnya.
    \end{itemize}
    \par Detail lengkap mengenai fungsionalitas setiap aktor dapat dilihat pada \textit{use case diagram} yang akan dijelaskan pada bagian perancangan sistem.
\end{enumerate}

\subsection{Perancangan Sistem}
\par Perancangan sistem adalah tahap penting dalam proses penelitian yang bertujuan untuk memastikan bahwa penelitian dapat dilaksanakan secara jelas, baik, dan terstruktur. Pada tahap ini, dilakukan desain prototipe aplikasi berdasarkan analisis kebutuhan yang telah dilakukan sebelumnya.

\begin{enumerate}
    \item \textbf{\textit{Use Case Diagram}}
    \par \textit{Use case diagram} merupakan salah satu komponen penting dalam \textit{Unified Modeling Language} (UML) yang digunakan untuk memodelkan interaksi antara aktor (pengguna) dan sistem berdasarkan fungsionalitas yang disediakan. Diagram ini menggambarkan skenario bagaimana pengguna berinteraksi dengan sistem melalui berbagai fitur, serta membantu dalam mengidentifikasi kebutuhan sistem secara fungsional. Dalam penelitian ini, sistem informasi pemesanan pada Bukhari Service Center memiliki dua aktor utama, yaitu pelanggan (\textit{customer}) dan admin, yang masing-masing memiliki hak akses dan aktivitas yang berbeda dalam sistem.
    \begin{enumerate}
        \item \textbf{\textit{Use Case Diagram} Pelanggan}
        \par \textit{Use case diagram} untuk pelanggan menggambarkan berbagai aktivitas yang dapat dilakukan oleh pengguna jasa. Pelanggan dapat melakukan registrasi akun dan \textit{login} ke dalam sistem. Setelah berhasil masuk, pelanggan dapat melihat daftar layanan dan harga \textit{sparepart}, melakukan pemesanan layanan servis, memilih jenis servis (datang ke toko atau memanggil teknisi ke lokasi), serta menentukan jadwal layanan yang diinginkan. Selain itu, pelanggan juga dapat memantau status perbaikan handphone secara \textit{real-time}, menerima notifikasi melalui email terkait progres servis dan biaya, melihat riwayat layanan sebelumnya, dan membatalkan pemesanan apabila diperlukan. Semua aktivitas tersebut dirancang untuk meningkatkan kenyamanan dan transparansi bagi pelanggan. Berikut \textit{Use Case Diagram} Pelanggan dapat dilihat pada Gambar \ref{fig:use-case-pelanggan}.
        \begin{figure}[H]
            \centering
            \includegraphics[width=1\linewidth]{assets/images/bab3/UsecasePelanggan.png}
            \caption{\textit{Use Case Diagram} Pelanggan}
            \label{fig:use-case-pelanggan}
        \end{figure}

        \item \textbf{\textit{Use Case Diagram} Admin}
        \par \textit{Use case diagram} untuk admin mencakup aktivitas yang berkaitan dengan pengelolaan sistem secara keseluruhan. Admin memiliki akses untuk \textit{login} ke sistem, mengelola data pelanggan (menambah, mengedit, dan menghapus akun), serta mengelola data layanan dan harga \textit{sparepart}. Admin juga bertanggung jawab dalam memverifikasi pemesanan yang masuk, memperbarui status servis, dan memastikan informasi yang ditampilkan kepada pelanggan tetap akurat dan terkini. Selain itu, admin dapat menghasilkan laporan transaksi dan layanan yang berguna untuk evaluasi operasional. Perancangan \textit{use case diagram} ini bertujuan untuk memastikan bahwa seluruh kebutuhan pengguna dapat diakomodasi secara baik oleh sistem. Berikut \textit{Use Case Diagram} Admin dapat dilihat pada Gambar \ref{fig:use-case-admin}.
        \begin{figure}[H]
            \centering
            \includegraphics[width=1\linewidth]{assets/images/bab3/UsecaseAdmin.png}
            \caption{\textit{Use Case Diagram} Admin}
            \label{fig:use-case-admin}
        \end{figure}
    \end{enumerate}

    \item \textbf{\textit{Entity Relationship Diagram} (ERD)} 
    \par \textit{Entity Relationship Diagram} (ERD) merupakan representasi visual yang digunakan untuk merancang struktur basis data dalam suatu sistem. Diagram ini menggambarkan entitas, atribut, serta relasi yang terjadi antar entitas secara sistematis. \textit{Entity Relationship Diagram} (ERD) dapat dilihat pada Gambar \ref{fig:ERD}.
    \begin{figure}[H]
        \centering
        \includegraphics[width=1\linewidth]{assets/images/bab3/ERDfixkali.png}
        \caption{\textit{Entity Relationship Diagram}}
        \label{fig:ERD}
    \end{figure}
    \begin{enumerate}
        \item Relasi antara Pengguna dan Service bersifat One-to-Many. Satu pengguna dapat mengajukan banyak permintaan service dari waktu ke waktu (mis. beberapa kali membawa HP untuk diperbaiki), tetapi setiap catatan service hanya diajukan oleh satu pengguna tertentu.
        \item Relasi antara Service dan Handphone pada diagram ini dinyatakan One-to-One. Artinya setiap record Service berfokus pada satu Handphone tertentu, yang mana setiap entri service menunjuk ke satu handphone yang sedang ditangani.
        \item Relasi antara Waktu dan Service adalah One-to-Many (satu Waktu — banyak Service). Satu entitas Waktu, misalnya satu slot jadwal/shift atau kategori waktu, bisa menjadi acuan untuk banyak service yang dijadwalkan pada slot atau kategori waktu yang sama.
        \item Relasi antara Handphone dan KendalaHandphone bersifat One-to-Many. Satu handphone dapat mempunyai beberapa kendala berbeda (misalkan baterai, LCD, tombol) yang tercatat, sedangkan setiap record KendalaHandphone merujuk pada satu handphone tertentu.
        \item Relasi antara KendalaHandphone dan PergantianBarang pada diagram ini ditunjukkan One-to-One. Setiap kendala yang tercatat dapat menghasilkan satu tindakan pergantian barang.
        \item Relasi antara entitas Kendala Perangkat dan Pergantian Barang ini bersifat One-to-One, setiap kendala perangkat menghasilkan satu tindakan pergantian barang yang spesifik, dan setiap pergantian barang tercatat sebagai akibat dari satu kendala tertentu.
    \end{enumerate}

    \item \textbf{\textit{Activity Diagram}}
    \par \textit{Activity Diagram} adalah representasi visual alur kerja dalam sistem, yang menggambarkan urutan aktivitas, keputusan, dan interaksi antar aktor. Diagram ini digunakan untuk memodelkan proses bisnis secara sistematis, menunjukkan langkah-langkah dari awal hingga akhir, mengidentifikasi titik keputusan (\textit{decision points}), dan memvisualisasikan kolaborasi antar peran. Berikut \textit{Activity Diagram} yang dapat dilihat pada Gambar \ref{fig:actDi}
    \begin{figure}[H]
        \centering
        \includegraphics[width=0.8\linewidth]{assets/images/bab3/ActivityDiagram.png}
        \caption{\textit{Activity Diagram} Proses Bisnis}
        \label{fig:actDi}
    \end{figure}
    \par Pada Gambar 3.5, \textit{activity diagram} sistem pemesanan layanan servis \textit{handphone} Bukhari Service Center menggambarkan alur kolaborasi terintegrasi antara Pelanggan dan Admin. Proses bisnis sistem dapat dijelaskan dalam beberapa tahapan berikut:
    
    \textbf{Tahap 1: Autentikasi Pengguna}
    \par Proses dimulai saat Pelanggan mengakses sistem dan melakukan autentikasi. Sistem memeriksa apakah pengguna sudah memiliki akun. Jika belum, pengguna diarahkan untuk melakukan registrasi dengan mengisi data seperti nama lengkap, email, nomor telepon, dan password. Setelah registrasi berhasil atau jika sudah memiliki akun, pengguna melakukan \textit{login} menggunakan email dan \textit{password}. Sistem memverifikasi kredensial, dan jika valid, pengguna diberikan akses ke halaman utama.
    
    \textbf{Tahap 2: Pembuatan Pemesanan Layanan}
    \par Setelah berhasil masuk, Pelanggan mengakses menu pemesanan layanan yang terdiri dari lima langkah berurutan:
    \begin{enumerate}
        \item \textbf{Input Informasi Perangkat}: Pelanggan memilih merek dan tipe \textit{handphone} yang akan di-\textit{service} dari daftar yang tersedia.
        \item \textbf{Deskripsi Masalah}: Pelanggan memilih jenis kerusakan dari kategori yang disediakan (misalnya: layar retak, baterai rusak, speaker bermasalah, dll.) dan dapat menambahkan catatan tambahan untuk menjelaskan masalah secara detail.
        \item \textbf{Pemilihan Jadwal}: Pelanggan memilih tanggal dan waktu layanan dari slot jadwal yang tersedia. Sistem menampilkan kalender dengan indikator slot yang sudah penuh atau masih tersedia.
        \item \textbf{Pilihan Jenis Layanan}: Pelanggan memilih antara dua opsi layanan, yaitu datang langsung ke toko Bukhari Service Center atau meminta teknisi datang ke lokasi pelanggan. Jika memilih layanan ke lokasi, pelanggan perlu mengisi alamat lengkap dan dapat menyertakan tautan Google Maps untuk mempermudah teknisi menemukan lokasi.
        \item \textbf{Konfirmasi Pemesanan}: Sistem menampilkan ringkasan lengkap pemesanan yang mencakup informasi perangkat, deskripsi masalah, jadwal layanan, jenis layanan, dan estimasi biaya jika tersedia. Pelanggan memeriksa kembali seluruh informasi dan dapat kembali ke langkah sebelumnya untuk melakukan perubahan jika diperlukan. Setelah memastikan semua data benar, pelanggan menekan tombol konfirmasi untuk mengirimkan pemesanan.
    \end{enumerate}
    
    \textbf{Tahap 3: Verifikasi oleh Admin}
    \par Setelah Pelanggan mengonfirmasi pemesanan, data pemesanan secara otomatis masuk ke sistem \textit{dashboard} Admin. Admin menerima notifikasi bahwa ada pemesanan baru yang perlu diverifikasi. Admin kemudian memeriksa kelengkapan dan validitas data pemesanan, termasuk informasi perangkat, jadwal yang diminta, dan ketersediaan teknisi. Pada titik keputusan verifikasi ini, terdapat dua kemungkinan:
    \begin{itemize}
        \item \textbf{Pemesanan Ditolak}: Jika terdapat masalah seperti data tidak lengkap, jadwal tidak tersedia, atau keterbatasan kapasitas, Admin dapat menolak pemesanan dengan memberikan alasan penolakan. Sistem secara otomatis mengirimkan notifikasi email kepada Pelanggan yang berisi informasi penolakan beserta alasannya. Pelanggan dapat membuat pemesanan baru dengan menyesuaikan informasi yang diperlukan. Proses selesai di sini untuk pemesanan yang ditolak.
        \item \textbf{Pemesanan Disetujui}: Jika semua data valid dan jadwal tersedia, Admin menyetujui pemesanan. Untuk layanan \textit{on-site} (teknisi ke lokasi), Admin mengatur penjadwalan teknisi yang akan bertugas. Setelah itu, Admin memperbarui status pemesanan menjadi "Diverifikasi" dan sistem mengirimkan notifikasi email konfirmasi kepada Pelanggan yang berisi detail pemesanan yang telah disetujui beserta informasi teknisi yang akan menangani (jika layanan \textit{on-site}).
    \end{itemize}
    
    \textbf{Tahap 4: Proses Perbaikan dan Pembaruan Status}
    \par Setelah pemesanan diverifikasi, proses perbaikan berlangsung dengan pembaruan status secara bertahap yang dilakukan oleh Admin berdasarkan laporan dari teknisi di lapangan. Alur pembaruan status sebagai berikut:
    \begin{enumerate}
        \item \textbf{Status "Menunggu Perbaikan"}: Status awal setelah verifikasi, menandakan perangkat siap untuk diperbaiki sesuai jadwal yang telah ditentukan.
        \item \textbf{Status "Sedang Diperbaiki"}: Admin memperbarui status ketika teknisi mulai melakukan diagnosis dan perbaikan perangkat. Pada tahap ini, teknisi mengidentifikasi komponen yang rusak dan melakukan perbaikan atau penggantian \textit{sparepart} yang diperlukan.
        \item \textbf{Status "Menunggu Konfirmasi Biaya"}: Jika diperlukan penggantian \textit{sparepart} atau biaya tambahan yang tidak terduga, Admin menghubungi Pelanggan untuk konfirmasi biaya. Sistem mengirimkan notifikasi email yang berisi rincian biaya \textit{sparepart} dan biaya jasa perbaikan. Pelanggan dapat menyetujui atau membatalkan perbaikan pada tahap ini.
        \item \textbf{Status "Selesai"}: Setelah perbaikan selesai dan perangkat telah diuji untuk memastikan berfungsi dengan baik, Admin memperbarui status menjadi "Selesai". Sistem mengirimkan notifikasi email kepada Pelanggan yang berisi informasi bahwa perangkat sudah siap diambil (untuk layanan di toko) atau akan diantarkan kembali (untuk layanan \textit{on-site}), disertai dengan rincian lengkap biaya akhir yang harus dibayarkan.
    \end{enumerate}
    
    \textbf{Tahap 5: Notifikasi dan Monitoring Real-Time}
    \par Sepanjang proses dari verifikasi hingga penyelesaian, Pelanggan menerima notifikasi otomatis melalui email setiap kali terjadi perubahan status. Notifikasi ini mencakup informasi detail tentang status terkini, estimasi waktu penyelesaian, dan rincian biaya jika ada pembaruan. Selain notifikasi email, Pelanggan juga dapat secara aktif memantau progres perbaikan secara \textit{real-time} melalui halaman "Lacak Status" di sistem. Halaman ini menampilkan \textit{timeline} visual yang menunjukkan tahapan mana yang sedang berlangsung, dilengkapi dengan informasi detail tentang perangkat, teknisi yang menangani, dan estimasi waktu penyelesaian.
    
    \textbf{Tahap 6: Riwayat Layanan}
    \par Setelah layanan selesai, seluruh informasi pemesanan tersimpan dalam riwayat layanan Pelanggan yang dapat diakses kapan saja melalui sistem. Riwayat ini mencakup informasi lengkap seperti tanggal layanan, jenis masalah, \textit{sparepart} yang diganti, biaya yang dikeluarkan, dan teknisi yang menangani. Fitur riwayat ini berguna bagi Pelanggan untuk melacak layanan sebelumnya dan dapat menjadi referensi untuk layanan berikutnya. Admin juga dapat mengakses riwayat ini untuk analisis kepuasan pelanggan dan evaluasi kinerja layanan.
    
    \par Keseluruhan proses ini dirancang untuk menciptakan siklus pelayanan yang efisien, transparan, dan terotomatisasi antara Pelanggan dan Admin melalui \textit{platform} berbasis \textit{website}. Sistem memastikan bahwa setiap langkah dalam proses pemesanan dan perbaikan terdokumentasi dengan baik, komunikasi antara Pelanggan dan Admin berjalan lancar melalui notifikasi otomatis, dan Pelanggan memiliki visibilitas penuh terhadap status perbaikan perangkat mereka tanpa harus menghubungi Admin secara manual.

    \item \textbf{\textit{High-fidelity Wireframe}}
    \par \textit{High-Fidelity Wireframe} merupakan prototipe antarmuka yang menyajikan desain sistem reservasi barbershop secara detail, lengkap dengan elemen visual seperti tombol, formulir input, dan menu navigasi. \textit{Wireframe} ini digunakan untuk menggambarkan tampilan nyata dari halaman reservasi, pemilihan tukang cukur, serta \textit{dashboard} admin beserta alur interaksinya. Selain itu, \textit{wireframe} ini berfungsi sebagai media validasi desain sebelum tahap pengembangan dimulai, guna memastikan bahwa rancangan antarmuka sudah sesuai dengan kebutuhan pengguna.
    \begin{enumerate}
        \item \textbf{Halaman Beranda} 
        \par Halaman beranda ini merupakan tampilan awal dari \textit{website} Bukhari Service Center yang dirancang dengan nuansa profesional melalui kombinasi warna biru dan putih yang bersih serta modern. Halaman ini menampilkan identitas \textit{brand} secara jelas melalui slogan, ajakan bertindak, dan penjelasan layanan unggulan seperti perbaikan ahli, pembaruan \textit{real-time}, serta layanan \textit{on-site}. Diperkuat dengan testimoni pelanggan dan daftar merek perangkat yang didukung, halaman ini tidak hanya memberikan kesan pertama yang kuat, tetapi juga membangun kepercayaan pengguna terhadap kualitas dan transparansi layanan yang ditawarkan. Halaman beranda dapat dilihat pada Gambar \ref{fig:halaman-beranda}.
        \begin{figure}[H]
            \centering
            \includegraphics[width=1\linewidth]{assets/images/bab3/HomePage.png}
            \caption{Halaman Beranda}
            \label{fig:halaman-beranda}
        \end{figure}

        \item \textbf{Halaman Daftar}
    \par Halaman daftar memungkinkan pengguna baru untuk membuat akun dengan mengisi data seperti nama lengkap, alamat email, nomor \textit{handphone}, \textit{password}, dan konfirmasi \textit{password}. Desain halaman ini konsisten dengan tema warna biru yang menenangkan, serta tata letak \textit{form} yang rapi dan mudah dipahami, sehingga pengguna dapat melakukan proses pendaftaran dengan cepat dan tanpa kebingungan. Halaman daftar dapat dilihat pada Gambar \ref{fig:halaman-daftar}.
    \begin{figure}[H]
        \centering
        \includegraphics[width=0.8\linewidth]{assets/images/bab3/daaftar.png}
        \caption{Halaman Daftar}
        \label{fig:halaman-daftar}
    \end{figure}

    \item \textbf{Halaman Masuk}
    \par Halaman masuk ini dirancang untuk memudahkan pengguna yang telah terdaftar dalam sistem untuk mengakses akun mereka. Dengan tampilan minimalis dan penggunaan gradasi warna biru yang lembut, halaman ini memberikan kesan profesional dan ramah pengguna. Pengguna hanya perlu memasukkan email dan \textit{password} untuk masuk. Halaman masuk dapat dilihat pada Gambar \ref{fig:halaman-masuk}.
    \begin{figure}[H]
        \centering
        \includegraphics[width=0.8\linewidth]{assets/images/bab3/masuk.png}
        \caption{Halaman Masuk}
        \label{fig:halaman-masuk}
    \end{figure}

    \item \textbf{Halaman Pemesanan Layanan}
    \par Halaman ini akan menampilkan tahapan-tahapan yang harus dilalui oleh pelanggan untuk melakukan pemesanan. Langkah pertama dalam proses pemesanan layanan di Bukhari Service Center yaitu meminta pelanggan untuk mengisi informasi dasar perangkat seperti merek dan tipe perangkat melalui \textit{checkbox} yang tersedia. Tampilan sederhana dengan indikator langkah 1 hingga 5 di bagian atas membantu pengguna memahami progres mereka dalam proses pemesanan. Tombol "Selanjutnya" di bagian bawah mengarahkan ke tahap berikutnya, memastikan alur pengisian data terstruktur dan efisien. Halaman untuk input informasi perangkat dapat dilihat pada Gambar \ref{fig:informasi-perangkat}.
    \begin{figure}[H]
        \centering
        \includegraphics[width=0.8\linewidth]{assets/images/bab3/step1.png}
        \caption{Halaman Informasi Perangkat}
        \label{fig:informasi-perangkat}
    \end{figure}

    \par Halaman kedua ini berfokus pada pendeskripsian masalah perangkat oleh pengguna. Pengguna dapat memilih topik kerusakan yang tersedia pada \textit{website} Bukhari Service Center seperti ganti baterai, layar depan retak, install ulang, speaker atau mikrofon bermasalah, tombol tidak berfungsi, atau kamera tidak berfungsi dengan baik. Navigasi "Kembali" dan "Selanjutnya" memungkinkan pengguna meninjau atau melanjutkan ke langkah penjadwalan, dengan indikator langkah 2 yang menandakan progres saat ini. Halaman untuk input deskripsi masalah \textit{handphone} pelanggan dapat dilihat pada Gambar \ref{fig:deskripsi-masalah}.
    \begin{figure}[H]
        \centering
        \includegraphics[width=0.8\linewidth]{assets/images/bab3/step2.png}
        \caption{Halaman Deskripsi Masalah}
        \label{fig:deskripsi-masalah}
    \end{figure}

\par Pada langkah ketiga, pengguna diminta menjadwalkan waktu layanan dengan memilih tanggal dan jam yang tersedia melalui kolom input tanggal dan waktu. Indikator langkah 3 menunjukkan posisi pengguna dalam alur lima tahap pemesanan. Tombol "Kembali" memungkinkan revisi informasi sebelumnya, sementara tombol "Selanjutnya" mengarah ke halaman berikutnya. Tampilan intuitif ini memastikan pengguna dapat menentukan jadwal servis tanpa kebingungan. Halaman untuk input jadwal service dapat dilihat pada Gambar \ref{fig:pilih-jadwal}.
    \begin{figure}[H]
        \centering
        \includegraphics[width=0.8\linewidth]{assets/images/bab3/step3.png}
        \caption{Halaman Pilih Jadwal}
        \label{fig:pilih-jadwal}
    \end{figure}

    \par Halaman keempat ini memperkenalkan pemilihan jenis layanan servis, di mana pengguna dapat memilih antara datang langsung ke Bukhari Service Center atau menggunakan layanan \textit{on-site} dengan mekanik yang datang ke lokasi pengguna. Untuk opsi layanan \textit{on-site}, terdapat kolom input untuk mencantumkan alamat detail dan opsi penggunaan tautan Google Maps. Indikator langkah 4 menunjukkan progres dalam alur lima tahap pemesanan. Tombol navigasi "Kembali" dan "Selanjutnya" memungkinkan penyesuaian sebelum konfirmasi akhir. Halaman untuk memilih layanan dapat dilihat pada Gambar \ref{fig:pilih-layanan}.
    \begin{figure}[H]
        \centering
        \includegraphics[width=0.8\linewidth]{assets/images/bab3/step4.png}
        \caption{Halaman Pilih Layanan}
        \label{fig:pilih-layanan}
    \end{figure}

    \par Halaman kelima sebagai tahap konfirmasi akhir kini mencakup informasi tambahan tentang jenis layanan yang dipilih. Ringkasan pemesanan menampilkan detail perangkat (merek dan tipe \textit{handphone}), deskripsi masalah, jadwal servis, serta jenis layanan. Pada halaman ini juga terdapat biaya perbaikan yang harus dibayar. Pengguna dapat memverifikasi seluruh informasi sebelum menekan tombol "Konfirmasi" atau kembali ke tahap sebelumnya melalui "Kembali", dengan indikator langkah 5 menandakan penyelesaian proses. Halaman untuk konfirmasi pemesanan dapat dapat dilihat pada Gambar \ref{fig:konfirmasi-pemesanan}.
    \begin{figure}[H]
        \centering
        \includegraphics[width=0.8\linewidth]{assets/images/bab3/BOOKING ROOM 55.png}
        \caption{Halaman Konfirmasi Pemesanan}
        \label{fig:konfirmasi-pemesanan}
    \end{figure}

    \item \textbf{{Halaman Riwayat Pemesanan}}
    \par Halaman ini menyediakan rekam jejak historis seluruh layanan pengguna, menampilkan informasi tiap riwayat seperti nama perangkat, jenis masalah, tanggal servis, jenis layanan, dan status akhir penyelesaian. Selain itu, terdapat juga informasi berupa ulasan yang pernah diberikan terhadap pelayanan Bukhari Service Center. Halaman riwayat pemesanan dapat dilihat pada Gambar \ref{fig:riwayat-pemesanan}.
    \begin{figure}[H]
        \centering
        \includegraphics[width=0.8\linewidth]{assets/images/bab3/History.png}
        \caption{Halaman Riwayat Pemesanan}
        \label{fig:riwayat-pemesanan}
    \end{figure}

    \item \textbf{{Halaman Lacak Status Pemesanan}}
    \par Halaman ini memungkinkan pengguna memantau \textit{real-time} progres perbaikan perangkatnya dengan menampilkan status terkini, detail masalah, jadwal layanan, dan lokasi servis. Indikator visual tiga tahap (Belum dikerjakan, Sedang dikerjakan, Selesai) secara intuitif menunjukkan perkembangan pekerjaan mekanik, meminimalkan kebutuhan konfirmasi manual. Halaman lacak status pemesanan dapat dilihat pada Gambar \ref{fig:lacak-status-pemesanan}.
    \begin{figure}[H]
        \centering
        \includegraphics[width=0.8\linewidth]{assets/images/bab3/Tracking.png}
        \caption{Halaman Lacak Status Pemesanan}
        \label{fig:lacak-status-pemesanan}
    \end{figure}
    
    \end{enumerate}
\end{enumerate} 

\subsection{Pengembangan Sistem}
\par Pada penelitian "Rancang-Bangun Sistem Informasi Pemesanan pada Bukhari Service Center Berbasis \textit{Website}", metode Agile diterapkan sebagai pendekatan dalam pengembangan sistem. Metode ini dipilih karena memiliki keunggulan dalam hal fleksibilitas, proses iteratif yang cepat, serta kemampuannya dalam merespons perubahan kebutuhan secara dinamis. Agile yang digunakan disesuaikan agar dapat dijalankan secara individu, sehingga proses pengembangan dapat dilakukan secara bertahap dan berkelanjutan. Dengan pendekatan ini, sistem dapat dikembangkan dan disempurnakan secara berkala sesuai dengan kebutuhan fungsional dan umpan balik yang diperoleh.
\begin{itemize}
    \item \textbf{Perencanaan}
    \par Pada tahap ini, dilakukan proses perencanaan dan analisis kebutuhan sistem dengan mengumpulkan informasi dari \textit{stakeholder}, yaitu pelanggan dan admin Bukhari Service Center. Hasil dari tahap ini berupa daftar kebutuhan sistem, baik fungsional maupun non-fungsional, yang menjadi dasar dalam proses pengembangan. Selain itu, disusun pula rancangan sistem berupa \textit{use case diagram}, \textit{entity relationship diagram} (ERD), serta prototipe \textit{high-fidelity} yang berfungsi sebagai panduan dalam membangun sistem informasi pemesanan berbasis web secara terstruktur.
    \item \textbf{Desain}
    \par Pada tahap ini, dilakukan perancangan arsitektur sistem dengan memanfaatkan teknologi Next.js sebagai antarmuka \textit{frontend} dan \textit{backend}, serta PostgreSQL sebagai sistem manajemen basis data. Proses perancangan mencakup pembuatan desain antarmuka pengguna, perancangan struktur basis data, serta penentuan API endpoints yang akan digunakan dalam komunikasi antara \textit{frontend} dan \textit{backend}. Seluruh desain dirancang dengan mempertimbangkan aspek skalabilitas, keamanan data, dan kemudahan penggunaan agar sistem dapat berjalan secara optimal dan responsif terhadap kebutuhan pengguna.
    \item \textbf{Implementasi}
    \par Pada tahap ini, dilakukan proses pengembangan sistem berdasarkan rancangan yang telah disusun sebelumnya. Implementasi dilakukan secara bertahap dengan menerapkan pendekatan iteratif, dimulai dari pengembangan fitur-fitur utama seperti pemesanan layanan, pengelolaan data pelanggan, penjadwalan teknisi, dan pelacakan status perbaikan. Selain itu, pada tahap ini juga diterapkan konsep \textit{Progressive Web App} (PWA) untuk meningkatkan aksesibilitas dan kinerja sistem, meliputi pembuatan \textit{file} manifest.json untuk mendukung instalasi aplikasi ke \textit{home screen}, konfigurasi \textit{service worker} untuk pengelolaan \textit{cache} dan penyediaan konten secara \textit{offline}, serta integrasi \textit{push notification} untuk memberikan pembaruan status perbaikan kepada pelanggan secara \textit{real-time}. Setiap fitur, termasuk komponen PWA, diuji segera setelah pengembangan untuk memastikan fungsinya sesuai kebutuhan sistem dan dapat berjalan optimal sesuai skenario penggunaan yang telah direncanakan.
    \item \textbf{Pengujian Sistem}
    \par Pada tahap ini, dilakukan pengujian sistem secara komprehensif dengan menggunakan metode Blackbox Testing untuk memverifikasi setiap fungsionalitas sistem tanpa memeriksa struktur internal kode, serta pengujian API untuk memastikan bahwa komunikasi antara \textit{frontend} dan \textit{backend} berjalan dengan benar setelah seluruh endpoint API selesai dibuat. Pengujian ini bertujuan memastikan bahwa seluruh fitur bekerja sesuai dengan kebutuhan pengguna dan memberikan pengalaman yang optimal. Hasil dari proses pengujian ini digunakan sebagai acuan dalam melakukan perbaikan dan penyempurnaan sistem sebelum diterapkan secara penuh.
    % \item \textbf{Pengujian}
    % \par Tahapan pengujian dan evaluasi merupakan bagian penting dalam proses pengembangan Sistem Informasi Pemesanan pada Bukhari Service Center. Pada tahap ini, penulis akan melaksanakan beberapa jenis pengujian untuk memastikan kualitas dan kelayakan sistem. Pengujian yang dilakukan meliputi:
    % \begin{enumerate}
    %     \item Blackbox Testing: Dilaksanakan melalui proses User Acceptance Testing (UAT) guna memastikan bahwa seluruh fitur sistem berjalan dengan baik dan sesuai dengan kebutuhan pengguna.
    %     \item Pengujian API: Dilakukan setelah seluruh endpoint API selesai dibuat, untuk memastikan bahwa komunikasi antara frontend dan backend berjalan dengan benar.
    %     \item User Experience Questionnaire (UEQ): Digunakan untuk menilai sejauh mana pengalaman pengguna terhadap antarmuka dan kemudahan penggunaan sistem setelah diuji secara langsung.
    % \end{enumerate}
\end{itemize}

\subsection{Pengujian dan Evaluasi}
\par Tahapan pengujian dan evaluasi merupakan bagian penting dalam proses pengembangan Sistem Informasi Pemesanan pada Bukhari Service Center. Pada tahap ini, penulis akan melaksanakan pengujian pengalaman pengguna berupa \textit{User Experience Questionnaire} (UEQ) untuk menilai sejauh mana pengalaman pengguna terhadap antarmuka dan kemudahan penggunaan sistem untuk memastikan kualitas dan kelayakan sistem. 




 

 